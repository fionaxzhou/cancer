% \documentclass[twoside,11pt]{article}
\documentclass[11pt]{article}

\usepackage{times,fullpage}
\usepackage{amsthm,amsfonts,amsmath,amssymb,color,float,graphicx,verbatim}
\usepackage{subfigure} 
\usepackage{natbib}
\usepackage{algorithm,algorithmic}
\newcommand{\theHalgorithm}{\arabic{algorithm}}
\usepackage{hyperref}

\usepackage{mathtools}
\usepackage{bbm}
\usepackage{csquotes}

\newtheorem{theorem}{Theorem}
\newtheorem{proposition}{Proposition}
\newtheorem{claim}{Claim}
\newtheorem{lemma}{Lemma}
\newtheorem{corollary}{Corollary}
%\newtheorem*{conditions}{Conditions}
\newtheorem{definition}{Definition}
\newtheorem{assumption}{Assumption}

%\renewenvironment{proof}[1][Proof: ]{\noindent \textbf{#1}}{\qed\medskip}
%\renewcommand{\baselinestretch}{2}

\newcommand{\pr}[1]{\mathbb{P}\left[ #1 \right]}
\newcommand{\Var}[1]{\text{Var}\left[ #1 \right]}
\newcommand{\floor}[1]{\lfloor #1 \rfloor}
\newcommand{\ceil}[1]{\left\lceil #1 \right\rceil}
\newcommand{\abs}[1]{\left| #1 \right|}
\newcommand{\one}{\mathbbm{1}}
\newcommand{\reals}{\mathbb{R}}
\newcommand{\E}{\mathbb{E}}
\newcommand{\LL}{\mathbb{L}}
\newcommand{\sign}{\mathrm{sign}}
\newcommand{\half}{\frac{1}{2}}
\newcommand{\argmin}[1]{\underset{#1}{\mathrm{argmin}}}
\newcommand{\argmax}[1]{\underset{#1}{\mathrm{argmax}}}
\newcommand{\summ}{\displaystyle \sum}
\newcommand{\intt}{\displaystyle\int}
\newcommand{\var}{\text{Var}}
\newcommand{\nchoosek}[2]{\left(\begin{array}{*{20}c}#1\\#2\end{array}\right)}

\newcommand{\rank}{\text{rank}}
\newcommand{\relu}[1]{\left[ #1 \right]_+}
\newcommand{\coleq}{\coloneqq}
\newcommand{\set}[1]{\left\lbrace#1\right\rbrace}
\newcommand{\bas}{\text{Bas}}
\newcommand{\p}[1]{\left( #1 \right)}
\newcommand{\pcc}[1]{\left[ #1 \right]}
\newcommand{\poc}[1]{\left( #1 \right]}
\newcommand{\pco}[1]{\left[ #1 \right)}
\newcommand{\bbr}{\mathbb{R}}
\newcommand{\bbn}{\mathbb{N}}
\newcommand{\bbq}{\mathbb{Q}}
\newcommand{\ita}{\mathit{a}}
\newcommand{\itu}{\mathit{u}}
\newcommand{\itv}{\mathit{v}}
\newcommand{\itr}{\mathit{r}}
\newcommand{\itk}{\mathit{k}}
\newcommand{\itm}{\mathit{m}}
\newcommand{\itn}{\mathit{n}}
\newcommand{\itw}{\mathit{w}}
\newcommand{\itx}{\mathit{x}}
\newcommand{\ity}{\mathit{y}}
\newcommand{\itz}{\mathit{z}}
\newcommand{\itp}{\mathit{p}}
\newcommand{\ba}{\mathbf{a}}
\newcommand{\be}{\mathbf{e}}
\newcommand{\bx}{\mathbf{x}}
\newcommand{\bw}{\mathbf{w}}
\newcommand{\bg}{\mathbf{g}}
\newcommand{\bb}{\mathbf{b}}
\newcommand{\bu}{\mathbf{u}}
\newcommand{\bv}{\mathbf{v}}
\newcommand{\bz}{\mathbf{z}}
\newcommand{\bc}{\mathbf{c}}
\newcommand{\bd}{\mathbf{d}}
\newcommand{\bh}{\mathbf{h}}
\newcommand{\by}{\mathbf{y}}
\newcommand{\bn}{\mathbf{n}}
\newcommand{\bs}{\mathbf{s}}
\newcommand{\bp}{\mathbf{p}}
\newcommand{\bq}{\mathbf{q}}
\newcommand{\bmu}{\boldsymbol{\mu}}
\newcommand{\balpha}{\boldsymbol{\alpha}}
\newcommand{\bbeta}{\boldsymbol{\beta}}
\newcommand{\btau}{\boldsymbol{\tau}}
\newcommand{\bxi}{\boldsymbol{\xi}}
\newcommand{\blambda}{\boldsymbol{\lambda}}
\newcommand{\bepsilon}{\boldsymbol{\epsilon}}
\newcommand{\bsigma}{\boldsymbol{\sigma}}
\newcommand{\btheta}{\boldsymbol{\theta}}
\newcommand{\bomega}{\boldsymbol{\omega}}
\newcommand{\Lcal}{\mathcal{L}}
\newcommand{\Ocal}{\mathcal{O}}
\newcommand{\Acal}{\mathcal{A}}
\newcommand{\Gcal}{\mathcal{G}}
\newcommand{\Ccal}{\mathcal{C}}
\newcommand{\Xcal}{\mathcal{X}}
\newcommand{\Jcal}{\mathcal{J}}
\newcommand{\Dcal}{\mathcal{D}}
\newcommand{\Fcal}{\mathcal{F}}
\newcommand{\Hcal}{\mathcal{H}}
\newcommand{\Rcal}{\mathcal{R}}
\newcommand{\Ncal}{\mathcal{N}}
\newcommand{\Scal}{\mathcal{S}}
\newcommand{\Pcal}{\mathcal{P}}
\newcommand{\Qcal}{\mathcal{Q}}
\newcommand{\Wcal}{\mathcal{W}}
\newcommand{\Ld}{\tilde{L}}
\newcommand{\uloss}{\ell^\star}
\newcommand{\loss}{\mathcal{L}}
\newcommand{\losst}{\ell_t}
\newcommand{\norm}[1]{\left\|#1\right\|}
\newcommand{\inner}[1]{\left\langle#1\right\rangle}
\renewcommand{\comment}[1]{\textcolor{red}{\textbf{#1}}}
\newcommand{\vol}{\texttt{Vol}}

\newtheorem{example}{Example}

\newcommand{\secref}[1]{Sec.~\ref{#1}}
\newcommand{\subsecref}[1]{Subsection~\ref{#1}}
\newcommand{\figref}[1]{Fig.~\ref{#1}}
\renewcommand{\eqref}[1]{Eq.~(\ref{#1})}
\newcommand{\lemref}[1]{Lemma~\ref{#1}}
\newcommand{\corollaryref}[1]{Corollary~\ref{#1}}
\newcommand{\thmref}[1]{Thm.~\ref{#1}}
\newcommand{\propref}[1]{Proposition~\ref{#1}}
\newcommand{\appref}[1]{Appendix~\ref{#1}}
\newcommand{\algref}[1]{Algorithm~\ref{#1}}

\newcommand{\note}[1]{\textcolor{red}{\textbf{#1}}}

\usepackage{hyperref}
\hypersetup{
  colorlinks   = true, %Colours links instead of ugly boxes
  urlcolor     = blue, %Colour for external hyperlinks
  linkcolor    = blue, %Colour of internal links
  citecolor   = red %Colour of citations
}


% \usepackage{jmlr2e}

% Definitions of handy macros can go here

% \newcommand{\dataset}{{\cal D}}
% \newcommand{\fracpartial}[2]{\frac{\partial #1}{\partial  #2}}

% Heading arguments are {volume}{year}{pages}{submitted}{published}{author-full-names}

% \jmlrheading{1}{2000}{1-48}{4/00}{10/00}{Chengzhe Yan}

% Short headings should be running head and authors last names

% \ShortHeadings{On Relationship Between Spins and Linear Regions}{Chengzhe Yan, Changshui Zhang}
% \firstpageno{1}

\begin{document}

\title{2016 LUNA Challenge Description Document}

% \author{\name Chengzhe Yan \email yancz12@mails.tsinghua.edu.cn\\
%        \addr Department of Automation\\
%        Tsinghua University\\
%        Haidian District, Beijing, China
%        \and
%        \name Changshui Zhang \email zcs@mail.tsinghua.edu.cn\\
%        \addr Department of Automation\\
%        Tsinghua University\\
%        Haidian District, Beijing, China}

% \author{Chengzhe Yan\\
%         yancz12@mails.tsinghua.edu.cn\\
%         Department of Automation\\
%         Tsinghua University\\
        % Haidian District, Beijing, China
       % \and
       % Changshui Zhang\\
       % zcs@mail.tsinghua.edu.cn\\
       % Department of Automation\\
       % Tsinghua University
       % }
       % Haidian District, Beijing, China}

% \editor{Leslie Pack Kaelbling}

\maketitle

\begin{abstract}%   <- trailing '%' for backward compatibility of .sty file
This document contains brief introduction to models we applied and the fully detailed experiment procedure description.
\end{abstract}

% \begin{keywords}
%   DNN, Expressiveness, Local Minimas
% \end{keywords}

\section{Introduction to Models}

We adopt deep neural network model into detection tasks, based on Overfeat\cite{sermanet2013overfeat}, TensorBox\cite{stewart2015end} with a specially designed loss function. For CNN substructure to extract features from raw image input, Google Inception V1\cite{szegedy2015going} is used. LSTM\cite{hochreiter1997long} is used for detection and merge object proposed by region proposing network.
This model is built and running using Tensorflow\cite{abadi2016tensorflow}.

Inception V1 are designed for ImageNet challenge which is highly complicated. With such complexity, it would be easily to overfit the data since the task is considerable simple. We fine-tune the structure and choose lower layers architecture for feature extractor. Also larger dropout ratio is used for better generalization. Adam\cite{kingma2014adam} is used for training with default learning rate and other parameters.

\section{Experiment Design Details}
We use the dataset provided by LUNA\cite{setio2016validation}. The dataset is divided into 10 subsets with annotations given. The procedures for training and validation are described below:
\begin{enumerate}
\item Slice all the data into $512\times512$ sized image noted as $I$, using annotations given in the dataset, and here we use $I_i$ for images set from the $i$th subset. And slices with annotations are noted as $A$ and $A_i$ for $i$th subset.
\item Perform CV of $i$ round using all sliced image with annotation $A$ \textbf{except} the $i$-th subset $A_i$, with \textbf{argumentation using rotation and flipping}.
\item After training of each round, model learned is tested and evaluate on the images sliced from CT file \textbf{without} annotations.
\item We use the same hyper-parameters for training on each subset. The only variable is the number of training epochs.
\item Model and results submitted is the best average result from LUNA evaluation scripts on all subsets of the \textbf{same training epoch cycles} .
\end{enumerate}



% \acks{We would like to acknowledge support for this project
% from the National Science Foundation. }


\newpage

% \appendix
% \section*{Appendix A.}
% \label{app:theorem}

% \vskip 0.2in
\bibliographystyle{abbrv}
\bibliography{ref}

\end{document}
